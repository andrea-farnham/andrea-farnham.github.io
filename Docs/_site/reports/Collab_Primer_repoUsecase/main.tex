
% template for CRS Primer articles
% inspired by looks of UKRN Primers https://www.ukrn.org/primers/
\documentclass[a4paper,11pt]{report}
\usepackage[default]{sourcesanspro} % nice sans serif font
\usepackage[T1]{fontenc}
\usepackage[utf8]{inputenc}
\usepackage[english]{babel} 
\usepackage{graphicx} % figures
\usepackage{amsmath, amssymb} % math
\usepackage{doi} % automatic doi-links
\usepackage[round]{natbib} % bibliography
\usepackage{booktabs} % nicer tables
\usepackage[dvipsnames]{xcolor}
\usepackage{pdflscape}
\usepackage[most]{tcolorbox} % color text box
\usepackage{todonotes}
\definecolor{swissRNblue}{RGB}{34,36,100}
\definecolor{linkblue}{RGB}{19,168,158}
\definecolor{linkgreen}{RGB}{27,183,117}

\usepackage{hyperref}
\usepackage{href-ul}
\renewcommand\url[1]{{\href{#1}{#1}}}
\hypersetup{
    colorlinks,
    citecolor=linkgreen,
    filecolor=black,
    linkcolor=linkgreen,
    urlcolor=linkgreen,
    urlbreak=true,
}
 
\usepackage{geometry}
\geometry{
  a4paper,
  total={170mm,257mm},
  left=25mm,
  right=25mm,
  top=25mm,
  bottom=25mm
}

\usepackage{sectsty}
\usepackage{titlesec}
\titlespacing*{\section}
{0pt}{3ex plus 1ex minus .2ex}{1ex plus .2ex}
\titlespacing*{\subsection}
{0pt}{3ex plus 1ex minus .2ex}{1ex plus .2ex}
\setlength\parindent{0pt}
\sectionfont{\color{swissRNblue}}
\subsectionfont{\color{swissRNblue}}
\renewcommand{\bibsection}{\section*{References}}
\begin{document}

% title and authors here
% -----------------------------------------------------------------------------
\begin{minipage}{\textwidth}
    % title
    {\Huge \textcolor{swissRNblue}{Archiving Data: A Cognitive Neuroscience use-case}} \\[0.5ex]
    % authors with ORCID
    \href{https://orcid.org/0000-0002-1857-8607}{Gorka Fraga González}\\
    \href{https://orcid.org/xxxx}{Andrew Clark}\\
    \href{https://orcid.org/xxxx}{Alexis Hervais-Adelman }\\    
    \href{https://orcid.org/xxxx}{Susanne Weber?}\\
    \href{https://orcid.org/xxxx}{Melanie Röthlisberger?}\\    
    \href{https://orcid.org/xxxx}{Leonard Held}\\
    \url{https://doi.org/10.5281/zenodo.XXXXXXX}
\end{minipage}%
\begin{minipage}{0.02\textwidth}
    \centering
    \includegraphics[height=2cm]{CRS.jpg}
\end{minipage}%


\section*{What is data archiving?}
Brief definition of what we understand as data archiving. See UZH ZB resources 

\section*{What is an Open Research Data repository?}
Here shortly specify what an Open Research Data repository is. Ref to web for searching in repositories database 

\section*{Why use an ORD repository?}
Explain the obvious. Highlight that the information is not necessarily public access but researchers may enquire /send an application to access the data (restricted access)

\section*{How do you archive data?}
 First see how much your data follows the FAIR principles -- Short recap and link to ZB resources on "How fair is your data' . 
 Find a suitable repository and see if there are particular requirements
 
 Proceed

This is better illustrated with an example of a project with a relatively complex data. 

--------

\section*{Use case: a neurolinguistics project} 
This primer describes a use case for the upload of a neurolinguistics project into an Open Research Data (ORD) repository. The reference project involves and Speech-in-noise comprehension task in which participants have to (...). The project's data involves electroencephalography and behavioral data, as well as auditory stimuli presented in the experiment. The repository is xxxx 

\subsection*{Think of a data structure system}

First search for any exising standards in the field. Here we follow BIDS (insert short descript), why ? (xxx)
Best to struct the project like this from start

\subsection*{Organize your data}
This is best done at the start of a project.

[Insert Folder tree of the project here]


\subsection*{Decide what needs to be archive}
Not everything must or should be archived. 

Schematic of minimum project elements
\begin{itemize}
    \item  Data
    \begin {itemize} 
        \item Raw
        \item Preview/thumbnails
        \item Derivatives
    \end{itemize}
    \item  Metadata    
    \begin {itemize} 
        \item .JSON 
        \item Tables         
    \end{itemize}
    \item  Code    
    \item  Documentation
    \begin {itemize} 
        \item Filenaming convention
        \item Metadata/data specifications
        \item Procedures       
        \item READMEs 
    \end{itemize}
\end{itemize} 

\subsection*{Selecting a repository}
What makes a good repository for my data ? In this case (...)

Workflow chart: File naming, organisation, metadata.

\section*{Archiving} 
Here Andre fills in  some details about how in this particular repo we need to proceed or additional files/struct required by it


\section*{Common Issues and Troubleshooting} 
Here I would describe issues with this project. e.g. in this case we don't have a problem of too large data set, the main problems can come from manipulations into the raw data (from source to 'raw', which is not really raw). (see knownIssues.txt file)


\section*{Take home message} 



\begin{tcolorbox}[enhanced,fit to height=10cm,
  colback=linkgreen!85!black!10!white,colframe=linkgreen!75!black,title= \Large Technical Box 1 - Archiving scripts for stimuli presentation ,
  drop fuzzy shadow,watermark color=white,watermark text=CRS]
  
Here issues with version control and scripts to run experiments (e.g., what files where presented, etc)   
\end{tcolorbox}

\begin{tcolorbox}[enhanced,fit to height=10cm,
  colback=linkgreen!85!black!10!white,colframe=linkgreen!75!black,title= \Large Technical Box 2 - Challenges of EEG experiments ,
  drop fuzzy shadow,watermark color=white,watermark text=CRS]

  Here Andrew inserts common operations done to raw EEG data 
\end{tcolorbox}


\section*{More information}

\bibliographystyle{apalikedoiurl}
\bibliography{bibliography}

% date and license
\vfill
\scriptsize
Version from \today. \\
This work is published under a Creative Commons Attribution 4.0 International license
(\href{https://creativecommons.org/licenses/by/4.0/}{CC-BY 4.0}). \\
Edited by xxx (\href{https://orcid.org/0000-0003-2779-320X}{xxxxORCID}) and 
xxx). \\

\end{document}
